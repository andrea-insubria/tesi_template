\documentclass[12pt,oneside]{amsbook}%,draft

\usepackage[italian]{babel}
\usepackage[utf8]{inputenc}
\usepackage{amsmath,amsthm}%
\usepackage{amsfonts}%
\usepackage{amssymb}%
\usepackage{mathrsfs}
\usepackage{graphicx,pdfsync}
\usepackage{ulem,color}

%----------------------------------------------------------
% Impaginazione ufficiale tesi
%\usepackage[a4paper,top=3cm,bottom=3.5cm,left=3.5cm,right=2.5cm]{geometry}
% impaginazione stampa bozze 
\usepackage[a4paper,top=2cm,bottom=2cm,textwidth=17cm]{geometry}

%,includehead,includefoot
\linespread{2}


%---------------------------------------------------------
% Gestione hyperlink
\usepackage[colorlinks=true,naturalnames=true,urlcolor=blue]{hyperref}

%% impostazioni per formattazione ambienti
\theoremstyle{plain}
\newtheorem{theorem}{Teorema}[chapter]
\newtheorem{lemma}{Lemma}[chapter]
\newtheorem{proposition}{Proposizione}[chapter]
\newtheorem{corollary}{Corollario}[chapter]
\newtheorem{fatto}{Fatto}

\newcounter{examples}

\theoremstyle{definition}
\newtheorem{definition}{Definizione}[chapter]
%\newtheorem{esempio}{Esempio}[examples]
\newtheorem{example}{Esempio}[chapter]

\theoremstyle{definition}
\newtheorem*{remark}{Osservazione}


%----------------------------------------------------------
\newcommand{\F}{\mathscr{F}}
\renewcommand{\P}{\mathbb{P}}


\begin{document}
\frontmatter
\begin {titlepage}
\begin{center}
UNIVERSIT\`A DEGLI STUDI DELL'INSUBRIA - COMO\\
Dipartimento di Scienza ed Alta Tecnologia\\
Corso di Laurea in Matematica
\end{center}
\vfill
\begin{figure}[h]
	\begin{center} 
		\scalebox{.2}{\includegraphics{logo}} 
	\end{center} 
\end{figure}
\vfill
\begin{center}
\LARGE{Xxxxxxx Xxxxxxx Xxxxxx Xxxxxxxxxx}
\end{center}
\vfill
\vfill
\vfill
\begin{flushleft}
Relatore: dott. Andrea MARTINELLI\\
Correlatore: 
\end{flushleft}
\vfill
\begin{flushright}
Tesi di Laurea di\\
XXXXXXXXXXXXXXXXXXXXX\\
Matricola n. XXXXXX
\end{flushright}
\vfill
\begin{center}
Anno Accademico 20xx--20xx
\end{center}
\end{titlepage}

\tableofcontents



\chapter*{Introduzione}

Lorem ipsum dolor sit amet, consectetur adipiscing elit. Donec eget tincidunt diam, sed gravida massa. Morbi ac ante pulvinar, pretium enim vehicula, consectetur risus. Class aptent taciti sociosqu ad litora torquent per conubia nostra, per inceptos himenaeos. Praesent quis diam porta, lobortis justo sed, elementum urna. Nullam ex lacus, efficitur ut sodales varius, volutpat in orci. Praesent semper facilisis nulla, ac tempor sapien dictum ac. Praesent non nisl nisl.

Donec fringilla rhoncus augue ac mattis. Mauris erat sapien, tincidunt sed laoreet ut, egestas non nisi. Praesent tristique rutrum dui, vel cursus leo pellentesque et. Etiam convallis nibh ut maximus molestie. Etiam magna tortor, ultrices et massa eget, volutpat facilisis ipsum. Cras id felis nec libero tempor malesuada. Proin commodo dui leo, aliquam convallis libero accumsan eu. Vivamus faucibus laoreet pharetra. Sed a ligula condimentum sem venenatis molestie eget ut mi. Vestibulum est ex, congue in erat sed, fermentum elementum magna. Quisque et dolor libero. Maecenas convallis magna eu consectetur consequat. Maecenas fringilla congue ante vitae rutrum. Suspendisse feugiat maximus eleifend. Aliquam feugiat nibh ultricies orci interdum tincidunt.

Fusce id mollis nisi. Phasellus finibus consectetur mi, id vulputate tellus consectetur eu. Sed molestie enim nisi, eu condimentum est lobortis sit amet. Quisque nibh quam, luctus vel porta sit amet, molestie in orci. In turpis ipsum, sagittis eget nunc ac, aliquet mollis enim. Vivamus ac odio at metus lacinia pulvinar. Donec id accumsan tortor, eget finibus ex.

Duis ultricies pellentesque dolor, vitae blandit lorem rutrum sed. Duis eu quam luctus, molestie mauris nec, ultricies nunc. Sed diam nibh, faucibus ut tortor vitae, tristique euismod nisl. Quisque blandit imperdiet libero, at malesuada metus dictum vel. Sed ut nulla eros. Pellentesque nec arcu a urna maximus euismod. Quisque at turpis quis ante dapibus elementum vitae eget tortor. Etiam quis lacus ac erat congue pellentesque sed vel turpis.

Aliquam nec pretium augue. Mauris a enim libero. Etiam molestie lacus sapien, egestas rutrum lorem cursus vel. Vestibulum ante ipsum primis in faucibus orci luctus et ultrices posuere cubilia curae; Duis malesuada condimentum leo, eu consectetur mauris consequat et. Donec eget massa sapien. Nulla at lectus vel arcu blandit sodales. Nulla pharetra elit vel turpis porta dictum. Sed venenatis fringilla dapibus.

In a orci sapien. Ut aliquet pellentesque tortor sit amet tempor. Sed imperdiet libero leo, eget porttitor ligula mollis ut. Etiam ut venenatis ipsum. Phasellus euismod justo nec vulputate pellentesque. Quisque non congue velit. Nunc blandit lorem vel nisi posuere, vitae sagittis dolor volutpat. Maecenas eu lorem sed nisi cursus viverra in quis sem. Nulla id augue malesuada, viverra urna ac, ullamcorper ligula. Suspendisse potenti. Ut imperdiet feugiat ante, eget mollis enim placerat ut. Vivamus sed augue pellentesque magna luctus sodales a eget nibh. Fusce convallis accumsan turpis quis cursus.




\mainmatter

\chapter{A}

\begin{figure}[htbp]
\begin{center}
\includegraphics{figures/image}
\caption{default}
\label{default}
\end{center}
\end{figure}


\include{tesi_data/B}
\chapter{C}

\section{One}
Lorem ipsum dolor sit amet, consectetur adipiscing elit. Nunc ac nisi urna. Sed gravida felis vitae sapien eleifend, non varius libero auctor. Vestibulum nec lorem ac est lacinia lobortis quis non enim. Aenean consequat quam purus, in aliquet neque malesuada vel. Vivamus cursus laoreet libero eget vestibulum. Vestibulum in blandit dolor. Quisque ut mauris nibh. Cras consectetur egestas orci a tristique. Integer a lacinia dolor, vel convallis nisi. Donec eu ipsum pharetra, porta lorem sit amet, tincidunt justo. Fusce pulvinar efficitur urna eu consequat. Nullam at ligula sit amet dui dictum pharetra et eget nibh.

\section{Two}
Donec ullamcorper, elit sed iaculis efficitur, dolor erat porttitor metus, a maximus erat velit in nisi. Curabitur gravida eget lacus vel fermentum. Quisque ac vulputate diam. Vivamus porta fringilla lacinia. Aenean vestibulum aliquet erat ac congue. Morbi vulputate cursus lacus, et sollicitudin sem suscipit vitae. Quisque vestibulum libero vel mi venenatis consequat. Pellentesque aliquam, massa et porta suscipit, turpis libero bibendum leo, vel pulvinar est lectus eget urna. Vivamus ornare augue interdum nulla bibendum ornare. Mauris sit amet commodo est. Ut id mollis ligula.

\section{Three}
Nunc varius nunc id nulla faucibus, vel scelerisque nunc ultricies. Etiam tincidunt leo vel odio pulvinar iaculis. Pellentesque habitant morbi tristique senectus et netus et malesuada fames ac turpis egestas. Integer volutpat molestie quam et tristique. Donec a mauris convallis, eleifend ex id, ultricies ipsum. Sed viverra, urna dignissim pellentesque molestie, dolor eros rutrum diam, in convallis velit turpis ac nibh. Nam feugiat dui vel pretium cursus. Donec condimentum eros erat. In rutrum, velit ut convallis mollis, tortor ante facilisis metus, sed gravida nibh nisl ut ligula.

\section{Four}
Aliquam feugiat finibus sem et euismod. Nam porta dui ex, sit amet fermentum felis hendrerit vel. Duis scelerisque eget tortor ut iaculis. Nullam pharetra tortor quis feugiat auctor. Phasellus dui lectus, iaculis nec lacus ac, pellentesque auctor felis. Fusce ut tristique justo, vitae sollicitudin metus. Fusce nec tristique risus, sed fringilla augue. In hac habitasse platea dictumst. Pellentesque sed nunc et justo consectetur finibus non a nunc. Vestibulum ante ipsum primis in faucibus orci luctus et ultrices posuere cubilia curae; Donec ultricies facilisis quam ac luctus. Quisque non dolor eget dui rutrum sodales. Pellentesque habitant morbi tristique senectus et netus et malesuada fames ac turpis egestas. Pellentesque neque tortor, ultrices et rhoncus a, iaculis eget mi. Interdum et malesuada fames ac ante ipsum primis in faucibus. Pellentesque a ultrices sem.

Phasellus fermentum porta urna. Etiam auctor mauris sed vulputate hendrerit. Nunc facilisis tempus efficitur. Maecenas ornare eros quis leo auctor viverra. Suspendisse aliquam congue aliquet. Nam interdum at elit eget efficitur. Vestibulum gravida augue enim, sit amet hendrerit orci pellentesque id. Nam leo nunc, vehicula vitae eros ut, maximus ultricies justo.





%\begin{thebibliography}{9}
%\end{thebibliography}





\end {document}
